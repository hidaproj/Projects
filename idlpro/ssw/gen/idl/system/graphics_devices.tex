\documentstyle[11pt]{article}

\setlength{\oddsidemargin}{0in}		%  1 inch (0 + 1) from left.
\setlength{\topmargin}{-0.5in}		%  1 inch (1.5 - 0.5) from top.
\setlength{\textwidth}{6.5in}		%  1 inch (8.5 - 1 - 6.5) from right.
\setlength{\textheight}{9in}		%  1 inch (11 - 1 - 9) from bottom.

\addtolength{\parskip}{0.5\baselineskip}

%  The environment centerpage centers text vertically on the page.  This is
%  useful within a titlepage environment.

\newenvironment{centerpage}{\mbox{} \protect\vspace*{\fill}}{
	\protect\vspace*{\fill} \mbox{} \protect\\ \mbox{}}

\begin{document}		
\bibliographystyle{plain}		% Define bibliography style.

\begin{titlepage}
\hrule
{\noindent\bf CORONAL DIAGNOSTIC SPECTROMETER}\\
\vspace{-0.7\baselineskip}\\
{\noindent\Huge\bf SoHO}
\vspace{2mm}
\hrule
\vspace{3mm}
\centerline{\bf CDS SOFTWARE NOTE No. 5}
\vspace{3mm}
\hrule
\noindent Version 1.1 \hfill 14 September 1994
\vspace{2mm}
\hrule

\begin{centerpage}

\begin{center}
{\Large\bf SERTS GRAPHICS DEVICES ROUTINES}\\
\mbox{}\\
W. Thompson\\
Applied Research Corporation\\
NASA Goddard Space Flight Center\\
Laboratory for Astronomy and Solar Physics\\
Code 682.1\\
Greenbelt, MD 20771, USA\\
\mbox{}\\
William.T.Thompson.1@gsfc.nasa.gov\\
pal::thompson
\end{center}

\end{centerpage}

\end{titlepage}

These routines form the part of the SERTS subroutine library pertaining to
switching back and forth between graphics devices.  The philosophy behind these
routines is to be able to switch from one device to another, and to be in the
same state as when last using that device.  That way one could do something
like the following:
\begin{enumerate}
\item
Start in X-windows modes.  Start a plot.
\item
Switch to PostScript mode.  Start a plot there as well.
\item
Switch back to X-windows mode.  Do an overplot.
\item
Switch back to PostScript mode.  Also do an overplot.
\end{enumerate}
and have all the plots work out correctly.  The same philosophy is extended to
switching back and forth between windows on the same device, and between
multiple plots in the same window.

\section{Switching between graphics devices}

The ability to switch back and forth between different graphics devices without
losing the state information for each device is accomplished through the
routine SETPLOT (without underscore).  This routine is called in the same way
as the standard IDL routine SET\_PLOT (with underscore), e.g.
\begin{quote}
\begin{verbatim}
SETPLOT,'PS'
\end{verbatim}
\end{quote}
The state of each graphics device is stored in an internal common block for
later retrieval.  SETPLOT will only work properly if it is used exclusively to
change between graphics devices.

There are a number of routines that serve as a front end to SETPLOT.  These are
\begin{quote}
\begin{tabular}{ll}
TEK		& Tektronix terminal\\
REGIS		& Regis terminal\\
SUNVIEW		& SunView\\
XWIN		& X-windows\\
WIN		& Microsoft Windows\\
PS		& PostScript plot file\\
\end{tabular}
\end{quote}
Most of these do nothing more than call SETPLOT and print a simple message to
the screen.  However, the PS routine provides some additional capabilities
described below---one should use the command PS instead of SET\_PLOT,'PS'.

Certain system variables that control the appearance of the plot can be defined
to be different depending on the graphics device.  These include
\begin{quote}
!P.CHARSIZE\\
!P.FONT\\
!P.COLOR\\
!P.BACKGROUND
\end{quote}
This means that users should be aware that changing the values for one device
does not automatically change it for another device.  Thus, for example, if one
is making plots both to an X-windows display, and to a PostScript file, and
wants to change the character size on both displays, the system variable
!P.CHARSIZE has to be changed twice, e.g.
\begin{quote}
\begin{verbatim}
!P.CHARSIZE = 2         ;Generate plot on X-windows display.
PLOT, A

PS                      ;Generate plot in PostScript file.
!P.CHARSIZE = 2
PLOT, A

XWIN                    ;Switch back to X-windows.
\end{verbatim}
\end{quote}
Once they have been set, however, SETPLOT will remember the settings
individually for each graphics device.

As well as switching between graphics devices, SETPLOT is also used to define
default values for several system variables (e.g. !P.FONT) as a function of
graphics device.  Users installing this software on their own system can modify
this section of the code to suit their own needs.

\section{Switching between windows}

The routine SETWINDOW can be used to switch between two or more already
existing windows in the same way that SETPLOT switches between devices.  Like
SETPLOT, one can switch back and forth between two separate graphs in each
window.  It is called in the same way as WSET, e.g.
\begin{quote}
\begin{verbatim}
SETWINDOW, 3
\end{verbatim}
\end{quote}

Creating a new window with the WINDOW command will also automatically switch to
that window.  In order to save the settings for the current window, it may be
necessary to call SETWINDOW before calling WINDOW, e.g.
\begin{quote}
\begin{verbatim}
SETWINDOW
WINDOW, 2
\end{verbatim}
\end{quote}
(Calling SETWINDOW without any parameters saves the settings for the current
window.)

\section{Switching between multiple plots on one screen}

One can also place multiple plots on the same screen, and switch between
them, using the routine SETVIEW.  The results are similar to using !P.MULTI,
but the process is different.  The major advantage of using SETVIEW instead of
!P.MULTI is the ability to switch back and forth between the various plots, and
still be able to use commands such as OPLOT, even if other plots have been
generated in between.

The command
\begin{quote}
\begin{verbatim}
SETVIEW, 1, 3, 2, 5
\end{verbatim}
\end{quote}
means that the next plot will be one in a series of plots---specifically, it
will be the first of three from the left, and the second of five from the top.

As an example, suppose that one wants to make three plots in a row of the
quantities A, B, and C.  You can accomplish this by doing something like the
following:
\begin{quote}
\begin{verbatim}
ERASE
SETVIEW, 1, 3
PLOT, A
SETVIEW, 2, 3
PLOT, B
SETVIEW, 3, 3
PLOT, C
\end{verbatim}
\end{quote}
Calling SETVIEW with non-trivial parameters, as shown in the example,
automatically disables the automatic erase ordinary generated by commands such
as PLOT.  Thus, the ERASE command is needed to ensure that one starts with a
blank screen.  The plots are shown as being made in order from left to right,
but actually could be produced in any order.

Once the plots have been generated as shown, the quantity B\_PRIME could be
overlaid on top of the plot of B through the commands
\begin{quote}
\begin{verbatim}
SETVIEW, 2, 3
OPLOT, B_PRIME
\end{verbatim}
\end{quote}

Calling SETVIEW without any parameters resets the graphics behavior back to the
default.

When switching between graphics devices, the view must be defined separately
for each graphics device, but the view will be preserved when switching back
and forth.  For example,
\begin{quote}
\begin{verbatim}
SETVIEW, 1, 3, 1, 5     ;Generate plot on X-windows display.
PLOT, A

PS                      ;Generate plot in PostScript file.
SETVIEW, 1, 3, 1, 5
PLOT, A

XWIN                    ;Switch back to X-windows.
\end{verbatim}
\end{quote}
The same holds true for switching between windows using SETWINDOW.

\section{PostScript support}

There are several routines specifically oriented toward generating and
processing PostScript output files.  These are
\begin{quote}
\begin{tabular}{ll}
PS		& Open a PostScript file, or switch to an already opened
		  file.\\
PSCLOSE		& Close a PostScript file.\\
PSPLOT		& Close a PostScript file, and send it to the printer.
\end{tabular}
\end{quote}

\subsection{Opening PostScript files}

The routine PS can be used to redirect graphics output to a PostScript file.
The name of the file can either be passed explicitly, e.g.
\begin{quote}
\begin{verbatim}
PS, 'myfile'
\end{verbatim}
\end{quote}
or the filename can be omitted and then the default name ``idl.ps'' is used.
If the filename is passed without any extension, as in the above example, then
the extension ``.ps'' is used.  Once the PostScript file is opened, subsequent
calls to PS (without any parameters) returns to the previously opened file,
e.g.
\begin{quote}
\begin{verbatim}
PLOT, A                 ;Plot on X-windows display

PS, 'myfile'            ;Open PostScript file myfile.ps
PLOT, A

XWIN                    ;Return to X-windows display
OPLOT, B

PS                      ;Return to myfile.ps
OPLOT, B
\end{verbatim}
\end{quote}

The default orientation for plots produced with the PS command is landscape
mode---i.e. the plot is oriented with the X-axis aligned with the long
dimension of the paper.  If portrait mode is desired instead, then the keyword
/PORTRAIT should be added to the command, e.g.
\begin{quote}
\begin{verbatim}
PS, /PORTRAIT
\end{verbatim}
\end{quote}
In this case, portrait mode means that not only is the X-axis aligned with the
short dimension of the paper, but also that the plot will be taller than it is
wide.  Using the PS routine, both the landscape and portrait orientations make
use of the entire page.  The /LANDSCAPE keyword is the inverse of /PORTRAIT.

Encapsulated PostScript files, as used in incorporating into \TeX\ and \LaTeX\
documents (and in other software) can be generated by using the /ENCAPSULATED
keyword, e.g.
\begin{quote}
\begin{verbatim}
PS, /ENCAPSULATED, 'myfile.eps'
\end{verbatim}
\end{quote}
An explicit filename is always required with the /ENCAPSULATED switch.

Color PostScript files can be generated with the /COLOR switch.  This can be
combined with any of the other switches already discussed: /LANDSCAPE,
/PORTRAIT, or /ENCAPSULATED.  Also, the COPY switch can be used to copy the
current color tables into the PostScript file.  Since a call to SETFLAG is
generated, the SERTS image display software is required to use the COPY switch.

It should be noted that calling PS with any of the above keywords, or with a
filename, will force IDL to open a new PostScript file regardless of whether
there's already one open or not.

\subsection{Closing and printing PostScript files}

PostScript files can be closed with the command
\begin{quote}
\begin{verbatim}
PSCLOSE
\end{verbatim}
\end{quote}
This will not only close the PostScript file, but will also automatically
redirect graphics output back to whatever device the user was using previously.
Alternately, the command
\begin{quote}
\begin{verbatim}
PSPLOT
\end{verbatim}
\end{quote}
will not only close the file, but send it to the printer as well.

There are several methods that PSPLOT uses to decide which printer to use.  One
way is to specify the queue directly through the QUEUE keyword, e.g.
\begin{quote}
\begin{verbatim}
PSPLOT, QUEUE='myprinter'
\end{verbatim}
\end{quote}
If that's not passed, then PSPLOT checks for the existence of the system
environment variable (VMS: logical name) PSLASER.  Finally, on Unix systems the
default print queue will be used if no other information is given---it is
assumed that this is almost always a PostScript printer.  However, on VMS
systems the print queue has to be explicitly defined.

When printing color PostScript files, the environment variable PSCOLOR is used
instead of PSLASER.

The /DELETE keyword tells PSPLOT to automatically delete the file.  Depending
on the operating system, this may occur either after the file has been queued
for printing, or not until it has actually been printed.

If PSPLOT is used to close the file as well as print it, then it ``remembers''
what the name of the file is, and whether it was opened as a color PostScript
file or not.  However, once the file is closed, then it ``forgets'' this
information, and reverts to the defaults (i.e., filename ``idl.ps'' and without
color).  In that case, this information would have to be passed explicitly,
e.g.
\begin{quote}
\begin{verbatim}
PSPLOT, 'myfile.ps'
PSPLOT, 'mycolorfile.ps', /COLOR
\end{verbatim}
\end{quote}

\section{Equal X and Y scales}

The routine SETSCALE can be used to force the X and Y axes of a plot to be the
same.  In other words, one can use SETSCALE to ensure that squares will appear
as squares, and circles will appear as circles, and not as rectangles and
ellipses.

For example, suppose that the arrays X and Y represent positions in some
cartesian coordinate system, and that they both have the same units (e.g.
kilometers).  The commands
\begin{quote}
\begin{verbatim}
SETSCALE, X, Y
PLOT, X, Y, PSYM=1
\end{verbatim}
\end{quote}
will plot the data so that the scale (e.g. kilometers/inch) will be the same in
both directions.  The corners of the plot will also be adjusted to fit the
range of the data unless the keyword /NOADJUST is used.

An optional way to use SETSCALE is to pass it the ranges explicitly.  For
example:
\begin{quote}
\begin{verbatim}
SETSCALE, XMIN, XMAX, YMIN, YMAX
\end{verbatim}
\end{quote}
Or, if one's doing a contour plot, then one can pass the array to be contoured
to SETSCALE:
\begin{quote}
\begin{verbatim}
SETSCALE, ARRAY
CONTOUR, ARRAY
\end{verbatim}
\end{quote}
Although if CONTOUR is to be called with the optional arrays for the
coordinates along the X and Y axes, then those arrays should be passed to
SETSCALE instead of the array to be contoured:
\begin{quote}
\begin{verbatim}
SETSCALE, X, Y
CONTOUR, ARRAY, X, Y
\end{verbatim}
\end{quote}

Calling SETSCALE by itself will reset the graphics behavior to the default.
It's suggested that this be done right after generating the plot---it won't
interfere with any overplotting commands.  The routines discussed in the
sections above, SETPLOT, SETWINDOW, and SETVIEW, will automatically do this
before taking any actions.

The behavior of SETSCALE can be fine-tuned with the !ASPECT system variable.
This variable describes the ratio of the height of a pixel over its width,
normally 1.  The SETPLOT routine will keep track of !ASPECT as a function of
graphics device.

\end{document}
